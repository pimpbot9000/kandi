% Tiivistelmät tehdään viimeiseksi. 
%
% Tiivistelmä kirjoitetaan käytetyllä kielellä (JOKO suomi TAI ruotsi)
% ja HALUTESSASI myös samansisältöisenä englanniksi.
%
% Avainsanomakejen lista pitää merkitä main.tex-tiedoston kohtaan \KEYWORDS.


%\begin{svabstract}
%  Ett abstrakt hit 
%%(\languagename)
%\end{svabstract}

\begin{enabstract}
Domain Name System (DNS) is a fundamental part of the Internet that allows resolving human-readable domain names into IP addresses. Due to the publicity of DNS records and the ease of access, DNS makes it possible for a penetration tester or an attacker to acquire information about the target network infrastructure using DNS queries.

This thesis discusses different methods of leveraging DNS queries to enumerate a target network infrastructure.
The different techniques and their operating principles will be reviewed.

In addition to DNS enumeration techniques, the thesis presents an overview of DNS, discusses how DNS has evolved from a security point of view, and gives an overview of recent developments in DNS. In addition, enumerating cloud services is briefly discussed.

The conclusion is that the best results in DNS enumeration are achieved using multiple different methods. During the review of security in DNS, it became evident that despite the effort put into DNS security, the security of DNS is still lacking. Some relatively old standardized security features are not yet widely adopted and, for encrypting DNS traffic, standards have existed only for a few years.

%(\languagename)
\end{enabstract}
