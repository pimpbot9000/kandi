% Tiivistelmät tehdään viimeiseksi. 
%
% Tiivistelmä kirjoitetaan käytetyllä kielellä (JOKO suomi TAI ruotsi)
% ja HALUTESSASI myös samansisältöisenä englanniksi.
%
% Avainsanojen lista pitää merkitä main.tex-tiedoston kohtaan \KEYWORDS.

\begin{fiabstract}
Domain name system (DNS) eli nimipalvelinjärjestelmä on oleellinen osa nykyistä Internetiä. Nimipalvelinjärjestelmä toisaalta mahdollistaa sen, että penetraatiotestaaja tai hyökkääjä pystyy kartoittamaan kohteen verkkoinfrastuktuurin nimipalvelinkyselyiden avulla, nimipalvelintietojen ollessa julkisia ja helposti saatavilla. 

Tutkielma käsittelee erilaisia menetelmiä kohteen verkkoinfrastuktuurin kartoittamiseksi nimenomaan nimipalvelinkyselyitä hyväksi käyttäen. Tutkielma on pääosin kirjallisuuskatsaus, jonka keinoin perehdytään erilaisten palvelukartoitustekniikoiden toimintaperiaatteisiin.

Palvelukartoitustekniikoiden lisäksi tutkielmassa esitellään nimipalvelinjärjestelmän toiminta lyhyesti. Lisäksi käsitellään sitä miten nimipalvelinjärjestelmä on kehittynyt ajan saatossa erityisesti tietoturvan näkökulmasta sekä millaisia muutoksia nimipalvelinjärjestelmässä on tapahtunut viimeaikoina. Tutkielman lopussa käsitellään lyhyesti pilvipalveluita palvukartoituksen näkökulmasta.

%
%Tiivistelmätekstiä tähän (\languagename). Huomaa, että tiivistelmä tehdään %vasta kun koko työ on muuten kirjoitettu.
\end{fiabstract}

%\begin{svabstract}
%  Ett abstrakt hit 
%%(\languagename)
%\end{svabstract}

\begin{enabstract}
Domain Name System (DNS) is an integral part of modern Internet which allows resolving domain names into IP addresses. Due to publicity of DNS records and their easy access, DNS makes it possible for a penetration tester or an attacker to acquire information about the target network infrastructure using DNS queries.

This thesis discusses different methods of using DNS queries or DNS enumeration to map the target network infrastructure. Research method is mainly literature review focusing on different techniques and their operating principles. 

In addition to DNS enumeration techniques the thesis presents an overview of DNS, discusses how DNS has evolved from a security point of view and gives a brief overview of recent developments in DNS. In the end enumerating cloud services is briefly discussed.
%(\languagename)
\end{enabstract}
